% Options for packages loaded elsewhere
\PassOptionsToPackage{unicode}{hyperref}
\PassOptionsToPackage{hyphens}{url}
%
\documentclass[
]{article}
\usepackage{amsmath,amssymb}
\usepackage{lmodern}
\usepackage{iftex}
\ifPDFTeX
  \usepackage[T1]{fontenc}
  \usepackage[utf8]{inputenc}
  \usepackage{textcomp} % provide euro and other symbols
\else % if luatex or xetex
  \usepackage{unicode-math}
  \defaultfontfeatures{Scale=MatchLowercase}
  \defaultfontfeatures[\rmfamily]{Ligatures=TeX,Scale=1}
\fi
% Use upquote if available, for straight quotes in verbatim environments
\IfFileExists{upquote.sty}{\usepackage{upquote}}{}
\IfFileExists{microtype.sty}{% use microtype if available
  \usepackage[]{microtype}
  \UseMicrotypeSet[protrusion]{basicmath} % disable protrusion for tt fonts
}{}
\makeatletter
\@ifundefined{KOMAClassName}{% if non-KOMA class
  \IfFileExists{parskip.sty}{%
    \usepackage{parskip}
  }{% else
    \setlength{\parindent}{0pt}
    \setlength{\parskip}{6pt plus 2pt minus 1pt}}
}{% if KOMA class
  \KOMAoptions{parskip=half}}
\makeatother
\usepackage{xcolor}
\IfFileExists{xurl.sty}{\usepackage{xurl}}{} % add URL line breaks if available
\IfFileExists{bookmark.sty}{\usepackage{bookmark}}{\usepackage{hyperref}}
\hypersetup{
  hidelinks,
  pdfcreator={LaTeX via pandoc}}
\urlstyle{same} % disable monospaced font for URLs
\usepackage{color}
\usepackage{fancyvrb}
\newcommand{\VerbBar}{|}
\newcommand{\VERB}{\Verb[commandchars=\\\{\}]}
\DefineVerbatimEnvironment{Highlighting}{Verbatim}{commandchars=\\\{\}}
% Add ',fontsize=\small' for more characters per line
\newenvironment{Shaded}{}{}
\newcommand{\AlertTok}[1]{\textcolor[rgb]{1.00,0.00,0.00}{\textbf{#1}}}
\newcommand{\AnnotationTok}[1]{\textcolor[rgb]{0.38,0.63,0.69}{\textbf{\textit{#1}}}}
\newcommand{\AttributeTok}[1]{\textcolor[rgb]{0.49,0.56,0.16}{#1}}
\newcommand{\BaseNTok}[1]{\textcolor[rgb]{0.25,0.63,0.44}{#1}}
\newcommand{\BuiltInTok}[1]{#1}
\newcommand{\CharTok}[1]{\textcolor[rgb]{0.25,0.44,0.63}{#1}}
\newcommand{\CommentTok}[1]{\textcolor[rgb]{0.38,0.63,0.69}{\textit{#1}}}
\newcommand{\CommentVarTok}[1]{\textcolor[rgb]{0.38,0.63,0.69}{\textbf{\textit{#1}}}}
\newcommand{\ConstantTok}[1]{\textcolor[rgb]{0.53,0.00,0.00}{#1}}
\newcommand{\ControlFlowTok}[1]{\textcolor[rgb]{0.00,0.44,0.13}{\textbf{#1}}}
\newcommand{\DataTypeTok}[1]{\textcolor[rgb]{0.56,0.13,0.00}{#1}}
\newcommand{\DecValTok}[1]{\textcolor[rgb]{0.25,0.63,0.44}{#1}}
\newcommand{\DocumentationTok}[1]{\textcolor[rgb]{0.73,0.13,0.13}{\textit{#1}}}
\newcommand{\ErrorTok}[1]{\textcolor[rgb]{1.00,0.00,0.00}{\textbf{#1}}}
\newcommand{\ExtensionTok}[1]{#1}
\newcommand{\FloatTok}[1]{\textcolor[rgb]{0.25,0.63,0.44}{#1}}
\newcommand{\FunctionTok}[1]{\textcolor[rgb]{0.02,0.16,0.49}{#1}}
\newcommand{\ImportTok}[1]{#1}
\newcommand{\InformationTok}[1]{\textcolor[rgb]{0.38,0.63,0.69}{\textbf{\textit{#1}}}}
\newcommand{\KeywordTok}[1]{\textcolor[rgb]{0.00,0.44,0.13}{\textbf{#1}}}
\newcommand{\NormalTok}[1]{#1}
\newcommand{\OperatorTok}[1]{\textcolor[rgb]{0.40,0.40,0.40}{#1}}
\newcommand{\OtherTok}[1]{\textcolor[rgb]{0.00,0.44,0.13}{#1}}
\newcommand{\PreprocessorTok}[1]{\textcolor[rgb]{0.74,0.48,0.00}{#1}}
\newcommand{\RegionMarkerTok}[1]{#1}
\newcommand{\SpecialCharTok}[1]{\textcolor[rgb]{0.25,0.44,0.63}{#1}}
\newcommand{\SpecialStringTok}[1]{\textcolor[rgb]{0.73,0.40,0.53}{#1}}
\newcommand{\StringTok}[1]{\textcolor[rgb]{0.25,0.44,0.63}{#1}}
\newcommand{\VariableTok}[1]{\textcolor[rgb]{0.10,0.09,0.49}{#1}}
\newcommand{\VerbatimStringTok}[1]{\textcolor[rgb]{0.25,0.44,0.63}{#1}}
\newcommand{\WarningTok}[1]{\textcolor[rgb]{0.38,0.63,0.69}{\textbf{\textit{#1}}}}
\usepackage{graphicx}
\makeatletter
\def\maxwidth{\ifdim\Gin@nat@width>\linewidth\linewidth\else\Gin@nat@width\fi}
\def\maxheight{\ifdim\Gin@nat@height>\textheight\textheight\else\Gin@nat@height\fi}
\makeatother
% Scale images if necessary, so that they will not overflow the page
% margins by default, and it is still possible to overwrite the defaults
% using explicit options in \includegraphics[width, height, ...]{}
\setkeys{Gin}{width=\maxwidth,height=\maxheight,keepaspectratio}
% Set default figure placement to htbp
\makeatletter
\def\fps@figure{htbp}
\makeatother
\setlength{\emergencystretch}{3em} % prevent overfull lines
\providecommand{\tightlist}{%
  \setlength{\itemsep}{0pt}\setlength{\parskip}{0pt}}
\setcounter{secnumdepth}{-\maxdimen} % remove section numbering
\ifLuaTeX
  \usepackage{selnolig}  % disable illegal ligatures
\fi

\author{}
\date{}

\begin{document}

\hypertarget{hw2}{%
\subparagraph{HW2}\label{hw2}}

\hypertarget{zhengmao-zhang}{%
\subparagraph{Zhengmao Zhang}\label{zhengmao-zhang}}

\hypertarget{problem-1-jarvis-march-gift-wrapping-algorithm}{%
\paragraph{Problem 1: Jarvis March (Gift Wrapping
Algorithm)}\label{problem-1-jarvis-march-gift-wrapping-algorithm}}

\hypertarget{a-10-points-give-pseudocode-describing-the-jarvis-march-algorithm-a-brief-description-of-how-it-works-and-explain-its-best-and-worst-case-efficiency}{%
\subparagraph{(a) {[}10 points{]} Give pseudocode describing the Jarvis
March algorithm, a brief description of how it works, and explain its
best and worst case
efficiency.}\label{a-10-points-give-pseudocode-describing-the-jarvis-march-algorithm-a-brief-description-of-how-it-works-and-explain-its-best-and-worst-case-efficiency}}

Answer:

\begin{Shaded}
\begin{Highlighting}[]
\NormalTok{algorithm Grahamscan(S) {-}\textgreater{} res is}
\NormalTok{	// S is the set of points S is S[x][y] set}
\NormalTok{	// res is the opint }
\NormalTok{	res = []}
\NormalTok{	n = S.size() // n is number of all points}
	
\NormalTok{	// in this part we need find the lowest point, OR}
\NormalTok{	// highest point}
\NormalTok{	leftstPoint = S[0] //its start point}
\NormalTok{	for s in all S do}
\NormalTok{		leftestPoint = lefterPoint(lowestPoint, s)}
	
\NormalTok{	// this part will triversal all point set}
\NormalTok{	firstPoint = leftestPoint}
\NormalTok{	secondPoint = NULL}
\NormalTok{	repeat:}
\NormalTok{		res.push(firstPoint)}
\NormalTok{		secondPoint = S[firstPoint.Next]}
\NormalTok{		for s in S do}
\NormalTok{			if counterClockWise(firstPoint, s, secondPoint) do}
\NormalTok{				secondPoint = s}
\NormalTok{		firstPoint = secondPoint}
\NormalTok{ until firstPoint == leftestPoint}
\NormalTok{// this function is to test if the point s is in the line (that point1 and point2 connect) left}
\NormalTok{function counterClockWise(point1, point s, point2) {-}\textgreater{} bool do}
\NormalTok{	return ((point2.x {-} point1.x) * (s.y {-} point1.y) {-} (point2.y {-} point1.y) * (c.x {-} point1.x)) \textless{} 0}
\end{Highlighting}
\end{Shaded}

\textbf{TIME COMPEXITY} \textbf{O(nh )} h is the convex hull size

Worst Case \textbf{Θ(n\^{}2)}

Best case \textbf{O(nh)} h is the convex hull

for the \textbf{best case}, the convex hull is a triangle, and only have
3 point on hull.

for the \textbf{worest case}, its all points are in solution hull.

\begin{figure}
\centering
\includegraphics{/Users/zhengmao/Library/Application Support/typora-user-images/image-20220116234735028.png}
\caption{}
\end{figure}

\hypertarget{b-5-points-give-an-example-input-on-which-jarvis-march-will-perform-significantly-better-than-grahams-scan-and-explain-why-it-will-perform-better}{%
\subparagraph{(b) {[}5{]} points Give an example input on which Jarvis
March will perform significantly better than Graham's scan and explain
why it will perform
better.}\label{b-5-points-give-an-example-input-on-which-jarvis-march-will-perform-significantly-better-than-grahams-scan-and-explain-why-it-will-perform-better}}

Answer:

Data Set: {[}{[}0,0{]} , {[}100, 0{]}, {[}0, 100{]},
{[}1,1{]},{[}2,2{]}, {[}3,3{]}, {[}4,4{]}\ldots\ldots{[}49,49{]},
{[}50,25{]}, {[}25,50{]}, {[}25,37.5{]}, {[}37.5,25{]}{]}

There are 3 lines in the data set, one is X=Y, another is
{[}50,0{]}-\textgreater{[}0, 100{]}, and {[}0,50{]} -\textgreater{}
{[}100,0{]}

for this data set, the convex hall is \{{[}0,0{]}, {[}100, 0{]},
{[}0,100{]}\}

for the \textbf{Jarvis March} , this data set is best case for Jarvis
March. the complelity is O(hn), h = 3. Its almost to O(n)

for the \textbf{Graham's scan},it is worest case for it, cuz there are 3
lines start at convex hall. So in the algorithm. the complexity is O(n
log n)

\hypertarget{c-5-points-give-an-example-input-on-which-grahams-scan-will-perform-significantly-better-than-jarvis-march-and-explain-why-it-will-perform-better}{%
\subparagraph{(c) {[}5{]} points Give an example input on which Graham's
Scan will perform significantly better than Jarvis March and explain why
it will perform
better.}\label{c-5-points-give-an-example-input-on-which-grahams-scan-will-perform-significantly-better-than-jarvis-march-and-explain-why-it-will-perform-better}}

Data set: {[}{[}0, 0{]}, {[}1,0{]}, {[}2,0{]}, {[}3,
0{]}\ldots\ldots{[}100,0{]}{]}, {[}, {[}0,1{]}, {[}0,2{]},
{[}0,3{]}\ldots\ldots{[}0,100{]}{]}, {[}{[}99, 1{]},{[}98, 2{]},
{[}97,3{]}, \ldots\ldots{[}1,99{]}{]}

In this data set, the convex hall is all points in this set.

for the \textbf{Jarvis March} , this data set is worst case for Jarvis
March. cuz the h is n, so the complexity is O(n\^{}2)

for the \textbf{Graham's scan},it is best case for it, cuz there are 3
lines start at convex hall. So in the algorithm. the complexity is O(n
log n)

\hypertarget{problem-2-find-the-missing-number}{%
\paragraph{Problem 2: Find the Missing
Number}\label{problem-2-find-the-missing-number}}

\hypertarget{a-5-points-give-an-efficient-algorithm-for-finding-the-missing-number-show-its-complexity-and-argue-its-correctness-you-should-try-for-on-time-less-efficient-solutions-will-still-get-partial-credit}{%
\subparagraph{(a) {[}5 points{]} Give an efficient algorithm for finding
the missing number, show its complexity, and argue its correctness. (You
should try for O(n)-time, less efficient solutions will still get
partial
credit)}\label{a-5-points-give-an-efficient-algorithm-for-finding-the-missing-number-show-its-complexity-and-argue-its-correctness-you-should-try-for-on-time-less-efficient-solutions-will-still-get-partial-credit}}

Answer:

if its a sorted list, we can just do a compare function to make a{[}i{]}
compare with i.

\begin{Shaded}
\begin{Highlighting}[]
\NormalTok{dataSet = []}
\NormalTok{algorithm getMissingNumber() is:}
\NormalTok{	// n is the size of arr DataSet}
\NormalTok{	n = dataSet.size}
\NormalTok{	for i from 0 to n {-} 1 do}
\NormalTok{  	if dataSet[i] is not i then}
\NormalTok{  		// attention: should return i, its not dataSet[i], cuz dataSet i is exist}
\NormalTok{  		return i}
	
\end{Highlighting}
\end{Shaded}

In this algorithm, its ofc \textbf{O(n)} cuz its only one loop from 0 to
n, so its \textbf{O(n)}

if its a \textbf{unsorted} list, we need get the sum of range n - 1

\begin{Shaded}
\begin{Highlighting}[]
\NormalTok{dataSet = []}
\NormalTok{algorithm getMissingNumber() is:}
\NormalTok{	// n is the size of arr dataSet}
\NormalTok{	n = dataSet.size}
\NormalTok{	sum = 0}
\NormalTok{	for i in range(0, n {-} 1) do}
\NormalTok{		sum += i}
\NormalTok{	for each dates in dataSet do}
\NormalTok{		sum {-}= dates}
\NormalTok{	return sum}
\end{Highlighting}
\end{Shaded}

Its \textbf{O(n)}, cuz the get sum part is \textbf{O(n)}, and the search
part is \textbf{O(n)}, so its \textbf{O(n)}

\#\#\#\#\#

Answer: Use XOR operator.

\begin{Shaded}
\begin{Highlighting}[]
\NormalTok{dataSet = []}
\NormalTok{algorithm getMissingNumber() is:}
\NormalTok{	// n is the size of arr dataSet}
\NormalTok{	n = dataSet.size}
	
\NormalTok{	x1 = dataSet[0]}
\NormalTok{	x2 = 1}
\NormalTok{	for i in range(0, n {-} 2) do}
\NormalTok{		x1 = x1 XOR dataSet[i]}
\NormalTok{	for i in range(1, n {-} 1) do}
\NormalTok{		x2 = x2 XOR dataSet[i]}
\NormalTok{	return x1 XOR x2}
	
\end{Highlighting}
\end{Shaded}

cuz XOR can delete repeat number.

\begin{quote}
like x1 = x2 XOR x3

if we wanna get the x2, we can just x1 XOR x3
\end{quote}

for complexity, TIME is O(n), NSPACE is O(1). Cuz for the 1st XOR loop,
its O(n), the 2nd same. So its O(n)

\hypertarget{b-10-points-for-this-question-you-are-not-allowed-to-access-an-entire-integer-with-a-single-operation-the-elements-of-the-list-are-represented-in-binary-and-the-only-operation-you-can-use-to-access-them-is-getbinarydigitaij-which-returns-the-jth-bit-of-element-ai-which-runs-in-constant-time-give-an-efficient-algorithm-for-finding-the-missing-number-under-these-constraints-show-its-complexity-and-argue-its-correctness-you-should-try-for-on-time-and-olog-n-space-less-efficient-solutions-will-still-get-partial-credit}{%
\subparagraph{(b) {[}10 points{]} For this question you are not allowed
to access an entire integer with a single operation. The elements of the
list are represented in binary, and the only operation you can use to
access them is GetBinaryDigit(A{[}i{]},j) which returns the jth bit of
element A{[}i{]} which runs in constant time. Give an efficient
algorithm for finding the missing number under these constraints, show
its complexity, and argue its correctness. (You should try for O(n)-time
and O(log n)-space, less efficient solutions will still get partial
credit)}\label{b-10-points-for-this-question-you-are-not-allowed-to-access-an-entire-integer-with-a-single-operation-the-elements-of-the-list-are-represented-in-binary-and-the-only-operation-you-can-use-to-access-them-is-getbinarydigitaij-which-returns-the-jth-bit-of-element-ai-which-runs-in-constant-time-give-an-efficient-algorithm-for-finding-the-missing-number-under-these-constraints-show-its-complexity-and-argue-its-correctness-you-should-try-for-on-time-and-olog-n-space-less-efficient-solutions-will-still-get-partial-credit}}

\hypertarget{example-if-we-run-getbinarydigitaij-with-ai--29-and-j--2-it-would-return-a-0-since-29--11101}{%
\subparagraph{Example: If we run GetBinaryDigit(A{[}i{]},j) with
A{[}i{]} = 29 and j = 2, it would return a 0 since 29 =
11101.}\label{example-if-we-run-getbinarydigitaij-with-ai--29-and-j--2-it-would-return-a-0-since-29--11101}}

Answer:

If the data set is a sorted

\begin{Shaded}
\begin{Highlighting}[]
\NormalTok{A = []}
\NormalTok{algorithm getMissingNumber() is:}
\NormalTok{	prelastdig = 1}
\NormalTok{	for i in range(n):}
\NormalTok{		if(GetBinaryDigit(A[i],2).lastdig == prelastdig)}
\NormalTok{			return A[i] {-} 1}
\end{Highlighting}
\end{Shaded}

If its not sorted.

for this problem, we can know, for every n size set, we get the unsigned
number num{[}i{]}'s length in binary, the max length of
nums{[}i{]}ToBi.toString.length is lg(n + 1) - 1. so we need find the
solution in this part.

\begin{Shaded}
\begin{Highlighting}[]
\NormalTok{algorithm getMissingNumber(subarr, n) is:}
\NormalTok{	if n is 1}
\NormalTok{		return 1 {-} subarr[0]}
\NormalTok{		leftsubarr = []}
\NormalTok{		rightsubarr = []}
\NormalTok{		highersite \textless{}{-} ceiling( lg(n+1) ) {-} 1}
\NormalTok{		for all num in subarr do}
\NormalTok{			if num[highersite] is 0 do}
\NormalTok{				leftsubarr.append(num)}
\NormalTok{			else}
\NormalTok{				rightsubarr.append(num)}
\NormalTok{		if leftsubset.size() \textless{} pow(highersite, 2)}
\NormalTok{			getMissingNumber(leftsubarr, pow(highersite, 2) {-} 1)}
\NormalTok{		else}
\NormalTok{			getMissingNumber(rightsubarr, n {-} pow(highersite, 2))}
\end{Highlighting}
\end{Shaded}

\textbf{PSPACE(LogN)}, cuz its a recurssion function. so we find they
always have the half size of current sub arr, so its run in Log n
\textbf{SPACE}

For the \textbf{PTIME(N)}, we can get the size of input is (2\^{}(log (n
+ 1) - 1)) - 1

then we can get \textbf{T(n)} \textless= T((2\^{}lg (n + 1)) - 1)

cuz 2\^{} lg(n + 1) ≈ n

T(n) = aO(n) - T(1) = \textbf{O(n)}

\end{document}
